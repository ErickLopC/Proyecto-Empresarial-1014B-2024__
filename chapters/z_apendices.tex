
%------------------------------------------------------------------
\chapter{Simulação Baseada em Eventos Discretos}\label{app:simulacao}
%------------------------------------------------------------------

In discrete-event simulation the system is controlled by events that change the state of the system. A discrete-event simulator has two main components: clock and event list. The clock controls the simulation time, while the event list maintains the list of active events. At any time (simulation time), there is only one event happening. 

The network aspects are simulated as follows. The messages from a node $n_i$ to a node $n_j$ are handled as a message deliver event. The event is schedule to occur in a time in future calculated based on the size of the message in bytes multiplied by the network speed. Thus, the computer behaves as a single instance of the entire system that may have thousands of nodes. For example, assume that we want to execute the plan where each MBR $m_i$ comes from a single node $n_i$. The processing starts at node $n_1$ that process the skyline and selects the filter points. The node $n_1$, than schedule two message deliver events, one to node $n_3$ and another to node $n_2$. The plan delivered to node $n_3$ is composed exclusively by the MBR $m_3$, while the plan delivered to node $n_2$ is composed by $m_2 \rightarrow m_4$. Since the message sent to node $n_3$ is smaller than the message sent to node $n_2$, the message sent to node $n_3$ will be scheduled to a shorter time in future. Thus, the next event to be processed will be message arrived in node $n_3$ that will start processing the skyline of $m_3$ using the filter points received from node $n_1$. If two messages are scheduled for the same simulation time, the simulator chooses any one of them randomly. 

The response time in our simulation is computed getting the maximum transfer time plus local process time. For example, if all transfer messages in the example above took 1 second and the local processing at each node also took 1 second, the response time will be defined by the response time of the longest path. Thus, the response time will be the time to send a message to $m_2$ plus the time to process the local skyline at $n_2$ plus the time to transfer the message to $n_4$  plus the time to compute the local skyline at $n_4$ plus the time to send the result back to $n_2$ plus the time to send the result back to $n_1$. 


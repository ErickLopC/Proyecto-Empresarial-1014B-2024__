\chapter{Importancia del impacto ambiental}


El objetivo de la evaluación del impacto ambiental es la sustentabilidad, pero para que un proyecto sea sustentable debe considerar además de la factibilidad económica y el beneficio social, el aprovechamiento razonable de los recursos naturales (Ver Criterios de Sustentabilidad).

Procedimiento de Evaluación de Impacto Ambiental (PEIA).
La evaluación de un estudio de impacto ambiental lo realiza la autoridad mediante un procedimiento de tipo técnico administrativo,  hay tres opciones mediante las cuales puede presentarse dependiendo del control que se tenga sobre los impactos y la magnitud del área donde se pretende desarrollar un proyecto:  

 a).-	 Informe preventivo
 b).-	 Manifestación de impacto ambiental modalidad particular y,
 c).-	 Manifestación de impacto ambiental modalidad regional.
a).- Informe preventivo

Requieren de presentar un Informe Preventivo y no una Manifestación de Impacto Ambiental en los siguientes casos:  

I.- Existan normas oficiales mexicanas u otras disposiciones que regulen las emisiones, las descargas, el aprovechamiento de recursos naturales y, en general, todos los impactos ambientales relevantes que puedan producir las obras o actividades;

II.- Las obras o actividades de que se trate estén expresamente previstas por un plan parcial de desarrollo urbano o de ordenamiento ecológico que haya sido evaluado por la Secretaría en los términos del artículo siguiente, o 

III.- Se trate de instalaciones ubicadas en parques industriales autorizados en los términos de la presente sección.

En los casos anteriores, la Secretaría, una vez analizado el informe preventivo, determinará, en un plazo no mayor de veinte días, si se requiere la presentación de una manifestación de impacto ambiental en alguna de las modalidades o si se está en alguno de los supuestos señalados.

b y c ).- Manifestación de Impacto Ambiental (MIA)

Se trata de un documento con base en estudios técnicos con el que las personas (físicas o morales) que desean realizar alguna de las obras o actividades previstas en el artículo 28 de la  LGEEPA, analizan y describen las condiciones ambientales anteriores a la realización del proyecto con la finalidad de evaluar  los impactos potenciales que la construcción y operación de dichas obras o la realización de las actividades podría causar al ambiente  y definir y proponer las medidas necesarias para prevenir, mitigar o compensar esas alteraciones.

\section{Objetivos 2030 que contribuye la empresa}

ODS 2: Hambre Cero

Contribución: Las cámaras climáticas pueden ser utilizadas para optimizar el crecimiento de flores comestibles y otras plantas, ayudando a mejorar la seguridad alimentaria y la nutrición en ciertas regiones.


ODS 8: Trabajo Decente y Crecimiento Económico

Contribución: Al proporcionar tecnología avanzada para la conservación y el cultivo de flores, la empresa puede fomentar el crecimiento económico en el sector agrícola y floricultor, creando empleos y mejorando las condiciones laborales.


ODS 9: Industria, Innovación e Infraestructura

Contribución: La empresa puede impulsar la innovación en la industria floricultora al introducir cámaras climáticas avanzadas, promoviendo la investigación y el desarrollo en tecnología agrícola.


ODS 12: Producción y Consumo Responsables

Contribución: Las cámaras climáticas ayudan a reducir el desperdicio de flores al optimizar su conservación y transporte, promoviendo prácticas de producción y consumo más sostenibles.


ODS 13: Acción por el Clima

Contribución: Al proporcionar soluciones para el control climático, la empresa contribuye a la adaptación y mitigación del cambio climático, permitiendo a los productores de flores reducir su huella de carbono mediante el uso eficiente de recursos.


ODS 15: Vida de Ecosistemas Terrestres

Contribución: Las cámaras climáticas pueden ayudar a preservar especies de flores raras y en peligro de extinción, promoviendo la biodiversidad y la conservación de los ecosistemas terrestres.


ODS 17: Alianzas para Lograr los Objetivos


Contribución: La empresa puede formar alianzas con organizaciones no gubernamentales, instituciones de investigación y otras empresas para promover el desarrollo sostenible en la floricultura y compartir conocimientos y tecnología.

Estas contribuciones muestran cómo una empresa dedicada a la venta de cámaras climáticas para flores puede alinear sus operaciones y objetivos con los ODS, promoviendo prácticas sostenibles y responsables en su industria.

\section{Aplicación de las leyes nacionales}


Como empresa nos aseguramos de cumplir con todas las regulaciones ambientales aplicables a su operación, lo cual incluye la obtención de permisos, la realización de evaluaciones de impacto ambiental, el control de emisiones y la gestión adecuada de residuos, entre otros aspectos. Para un cumplimiento adecuado, es recomendable contar con asesoría legal y ambiental especializada.

\subsection{Ley General de Equilibrio Ecológico y la Protección al Ambiente}

La LGEEPA establece que cualquier proyecto que pueda causar desequilibrio ecológico significativo debe someterse a una Evaluación de Impacto Ambiental (EIA). Esto incluye la instalación y operación de cámaras climáticas, ya que pueden tener implicaciones sobre el uso de energía, generación de residuos y emisiones atmosféricas.

Emisiones Atmosféricas: La empresa debe cumplir con las normas oficiales mexicanas (NOM) que regulan las emisiones de contaminantes al aire. Es posible que las cámaras climáticas utilicen sistemas de refrigeración y climatización que generen emisiones.

Residuos: La generación y manejo de residuos peligrosos y no peligrosos están regulados por la LGEEPA y otras leyes como la Ley General para la Prevención y Gestión Integral de los Residuos (LGPGIR).

Eficiencia Energética: Las cámaras climáticas deben diseñarse y operarse de manera que optimicen el consumo de energía. La NOM-ENER y otras regulaciones energéticas pueden ser aplicables.

Consumo de Agua: Si las cámaras utilizan agua para la humidificación o refrigeración, deben implementarse prácticas para su uso eficiente y cumplir con las normativas de la Comisión Nacional del Agua (CONAGUA).

La empresa es responsable de cualquier daño ambiental que pueda causar y debe implementar medidas de mitigación, restauración y compensación según lo establecido por la LGEEPA.

\subsection{Ley Federal de Responsabilidad Ambiental}

La empresa debe realizar una evaluación de impacto ambiental (EIA) para identificar, prevenir y mitigar posibles impactos ambientales negativos derivados de sus actividades. Esto es especialmente relevante si la instalación de las cámaras climáticas implica la modificación de ecosistemas naturales o la emisión de contaminantes.

Debe implementar medidas para prevenir y controlar la contaminación del aire, agua y suelo. La LFRA obliga a las empresas a adoptar tecnologías limpias y prácticas sostenibles para minimizar su huella ambiental.

En caso de que se produzca un daño ambiental, la empresa está obligada a repararlo. La LFRA establece mecanismos para que las autoridades ambientales exijan la reparación del daño y, en su caso, impongan sanciones.

La aplicación de la LFRA y otras leyes ambientales es esencial para garantizar que las operaciones de la empresa no dañen el entorno y para evitar sanciones legales.

\subsection{Ley General para la Prevención y Gestión Integral de los Residuos}

Informes de Cumplimiento Ambiental: La empresa deberá presentar informes periódicos a la Secretaría de Medio Ambiente y Recursos Naturales (SEMARNAT) y a otras autoridades competentes sobre el cumplimiento de las normas y regulaciones ambientales.

NOM-052-SEMARNAT-2005: Identificación y listado de los residuos peligrosos.

NOM-059-SEMARNAT-2010: Protección ambiental-especies nativas de México de flora y fauna silvestres-categorías de riesgo y especificaciones para su inclusión, exclusión o cambio-lista de especies en riesgo.

NOM-138-SEMARNAT/SSA1-2012: Límites máximos permisibles de hidrocarburos en suelos y las especificaciones para su caracterización y remediación.

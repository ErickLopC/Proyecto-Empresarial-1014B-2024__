%% ------------------------------------------------------------------------- %%
\chapter{Análisis de resultados}
\label{aresultados}
En este capítulo se analizarán los resultados obtenidos hasta ahora, si bien no se ha terminado de culminar el proyecto, se mostrarán algunos de los resultados preliminares, tomando estos resultados como retroalimentación para mejorar los sistemas.

\begin{figure}[H]
    \centering	
    \includegraphics[width=.8\textwidth]{img/Figure_1.png} 
    \caption{Gráfica de estabilidad del sistema, ángulo deseado vs tiempo}
	\label{fig:simulacionControlPython}
\end{figure}

La figura  \ref{fig:simulacionControlPython} muestra una gráfica obtenida de un programa que se ha realizado en Python en conjunto con la interfaz (figura \ref{fig:simulacionControlPython2}), en este programa se ingresa un ángulo deseado, el cual es enviado a través del puerto serial a la unidad de procesamiento, en este caso es un ESP32, una vez enviado el ángulo este comienza a activar los motores a diferentes velocidades con el objetivo de estabilizar el sistema en el ángulo proporcionado. Como se observa en la imagen, el algoritmo de control no cumple con su objetivo, ya que simplemente oscila y no logra mantener la estabilidad dada en el ángulo dado.

\begin{figure}[H]
    \centering	
    \includegraphics[width=.8\textwidth]{img/Control.jpg} 
    \caption{Programa de Python con interfaz y gráfica de estabilidad en tiempo real}
	\label{fig:simulacionControlPython2}
\end{figure}

Esto sucede ya que el control fuzzy implementado contempla valores de 0 a 255, que son parámetros del ancho de pulso que requiere el puente H de los motores, sin embargo los motores solo se activan hasta tener un ancho de pulso mayor a 150, por lo que es necesario cambiar de controlador, que englobe valores más altos y ya no menores al valor dicho.
Estos cambios se agregarán más adelante y se compararán con estos, para tener una mejor idea de como ha mejorado el controlador.

\begin{figure}[H]
    \centering	
    \includegraphics[width=.8\textwidth]{img/Rplot.png} 
    \caption{ Gráfica de estabilidad del sistema final}
	\label{fig:simulacionControlR}
\end{figure}

%------------------------------------------------------
\newpage
\section{Análisis de costos} 
En esta sección se analizarán los gastos directos, indirectos, la amortización así como el precio total del proyecto.
%------------------------------------------------------

\textbf{Costos directos}

\begin{table}[H]
\centering
\caption{Costos directos}
\begin{tabular}{p{4cm}|p{4cm}|p{4cm}}
\hline
\textbf{Componentes / Insumos }& \textbf{Cantidad} & \textbf{Costos} \\
\hline
Cabina de unicel&1&\$190\\
\hline
Escudo térmico, insulation&1 rollo&\$470\\
\hline
Filamento fibra de carbono&1&\$700\\
\hline
ESP32&2&\$300\\
\hline
Sensor de presión barométrica y temperatura&1&\$50\\
\hline
Placas de baquelita de 10x10 cm&1 paquete&\$160\\
\hline
Giroscopio MPU6050&1&\$65\\
\hline
Modulo GPS&1&\$200\\
\hline
Regulador de voltaje&1&\$45\\
\hline
Módulo GSM&1&\$140\\
\hline
Bateria lipo, 3s 1600mAh&1&\$700\\
\hline
Sensor humedad&1&\$165\\
\hline
Driver DRV8833&1&\$140\\
\hline
Hélices par&1&\$200\\
\hline
Cables estañados, calibre 22&1&\$220\\
\hline
Conectores JST-XH 2.0 mm&1&\$100\\
\hline
\textbf{Total:} & &\textbf{3845}\\

\hline
\end{tabular}
\label{tabla:costosDirectos}
\end{table}

%-----------------------------------------------------

\textbf{Costos indirectos}

\begin{table}[H]
\centering
\caption{Costos indirectos}
\begin{tabular}{p{6cm}|p{4cm}}
\hline
\textbf{Servicios } & \textbf{Costo} \\
\hline
Renta de servicio de luz &\$600\\
\hline
Insumos de alimentos&\$2000\\
\hline
Renta de internet&\$800\\

\hline
\end{tabular}
\label{tabla:costosIndirectos}
\end{table}

%-----------------------------------------------------
\textbf{Amortización}

La amortización es un concepto contable que hace referencia a la pérdida de valor de un bien durante su vida útil. Se registra con la intención de reflejar el valor real de una propiedad, considerando el uso, desgaste y valor residual. 

\begin{figure}[H]
    \centering	
    \includegraphics[width=.9\textwidth]{img/amortizacion.jpg} 
    \caption{ Amortización del producto}
	\label{fig:amortizacion}
\end{figure}

%-----------------------------------------------------

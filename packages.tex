\sloppy

\usepackage[T1]{fontenc} %write accents
\usepackage[utf8]{inputenc} %permits to write the text with accents
% \usepackage[latin1]{inputenc}
\usepackage[spanish]{babel}
%\usepackage{csquotes}
\usepackage[backend=biber, style=ieee]{biblatex}
\addbibresource{referencias.bib}
\addto\captionsspanish{\renewcommand{\tablename}{Tabla}}
\usepackage{adjustbox} 
\usepackage{changepage} %Margin adjustment and detection of odd/even pages
\usepackage{caption}
\usepackage{float}
%paquete de simbulo de marca registrada
\usepackage{textcomp}
%Paquete para códigos de programacion
\usepackage{listings}

%To prepare list of symbols
\usepackage{longtable}
%para colorear celdas de tablas
\usepackage[table]{xcolor}% http://ctan.org/pkg/xcolor
%Ajustar titulos
\usepackage{titlesec}
% Ajustes para el espaciado antes y después de los capítulos
\titlespacing{\chapter}{0pt}{\parskip}{\parskip}
% Personalizar el formato de los capítulos
\titleformat{\chapter}[display]
  {\normalfont\huge\bfseries}{}{0pt}{\Huge}




% Sets the margins of the document.
\oddsidemargin 5mm
\evensidemargin 5mm
\textwidth 150mm
\topmargin 0mm
\headheight 0mm
\textheight 225mm

% Selects font encoding
% \usepackage[T1]{fontenc}
\usepackage{ae,aecompl}
%\usepackage{times}

\usepackage{lipsum}

\usepackage{epigraph}
\setlength{\epigraphrule}{0pt}
\setlength{\afterepigraphskip}{2\baselineskip}

%header displays information according to document class and page number top right.
\usepackage{fancyhdr}


% Starts new paragraphs without indentation but with some space between the new and the previous paragraph.
\usepackage{parskip}
\setlength{\parskip}{1.5ex plus 0.4ex minus 0.4ex}
%\usepackage{indentfirst}

% try to keep paragraphs together
\widowpenalty=300
\clubpenalty=300

% Used to create tables with rows/cols spanning over se
\usepackage{array}
\newcolumntype{L}[1]{>{\arraybackslash}p{#1cm}}
\newcolumntype{C}[1]{>{\centering\arraybackslash}p{#1cm}}
\usepackage{multirow}

%Professional tables
\usepackage{booktabs}

% Math
\usepackage{amssymb}
\usepackage{amsmath}

% Used to include images with the includegraphics command
\usepackage{epsfig,subfigure,amstext}
\usepackage{graphicx}

\usepackage{makeidx}
\makeindex

% use for landscape pages
\usepackage{pdflscape}

%permits relative paths in the imported files
\usepackage{import}

%permits to create different environments
\usepackage{amsthm}

\newtheoremstyle{simple}
   {8pt}% hSpace above
   {}% hSpace below
   {\it}% hBody font
   {1em}% hIndent amount
   {}% hTheorem head font
   {\textup{:}}% hPunctuation after theorem head
   {.7em}% hSpace after theorem head
   {}% hTheorem head spec (can be left empty, meaning `normal')

\newtheoremstyle{enhanced}
   {8pt}% hSpace above
   {}% hSpace below
   {}% hBody font
   {1em}% hIndent amount
   {\itshape}% hTheorem head font
   {\textup{:}}% hPunctuation after theorem head
   {.7em}% hSpace after theorem head
   {}% hTheorem head spec (can be left empty, meaning `normal')


%links in pdf (index, references, and figures..., change the colors to black)
\usepackage[pdfborder={0 0 0},breaklinks=true,bookmarksopen=true,bookmarksopenlevel=1]{hyperref}


% Used to include algorithms
%\usepackage{algorithm,algorithmic}
%\renewcommand{\algorithmiccomment}[1]{\hfill \textit{//#1}}
%\newcommand{\vect}[1]{\overrightarrow{#1}}